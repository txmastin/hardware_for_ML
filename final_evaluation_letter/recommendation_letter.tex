\documentclass[12pt]{letter}
\usepackage[margin=1.5in]{geometry}
\usepackage{times}

\begin{document}

To whom it may concern,

It is a pleasure to write this recommendation for Tucker Mastin. I have worked with Tucker since September 2024 as his PhD advisor, and had the opportunity to work with him more formally during Spring 2025 when he enrolled in my graduate course \textit{ECE 510: Hardware for AI/ML}. Based on his contributions throughout the course, including a novel final project, engagement during lectures, and strong midterm and final assessment performances, I recommend that Tucker receive a final grade of \textbf{A}.

Tucker’s final project focused on the design, simulation, and benchmarking of a memcapacitive Leaky Integrate-and-Fire (LIF) neuron, implemented in SPICE using a Biolek-model memcapacitor. The circuit implements biologically motivated neuronal dynamics using readily available analog components, making it suitable for benchtop experimentation. The neuron design includes upward spike rate adaptation, continuous-time analog integration, and a Schmitt-trigger-based spike output with a MOSFET reset mechanism. Notably, it supports stable operation at up to 72kHz and targets sub-100pJ energy per spike.

The technical motivation behind his project is well-founded and clearly motivated: current neuromorphic platforms largely rely on digital approximations or custom ASICs, which are inaccessible for physical prototyping or interfacing with experimental materials, such as physically implemented, cutting-edge memcapacitors. Tucker’s design directly addresses this gap by offering a practical and testable circuit that could serve as a platform for embedded AI research and experimental memcapacitor validation. The project is especially relevant in the context of resource-constrained edge applications such as control systems, robotics, etc., where dense dynamics per unit and energy efficiency are critical.

His accompanying documentation is thorough and well-structured. It includes a theoretical overview of fractional dynamics and the relevant neuroscience motivations, implementation-level circuit descriptions, simulation details, and a comparative analysis of existing hardware paradigms (e.g., CPUs, GPUs, Loihi, FPGAs). The circuit stands out in its combination of biological plausibility, computational richness, and real-world applicability, traits not often found in benchtop-compatible neuromorphic designs.

Outside the final project, Tucker completed 17 of the course challenges and participated in class discussions. Furthermore, he engaged with his peers outside of class, discussing course materials, providing feedback, sharing insights, and advising on coursework. His midterm and final assessments were completed at a high level and reflect a clear grasp of the material. 

In summary, Tucker consistently demonstrated technical ability, independent initiative, and a clear understanding of the biological and computational goals of his project. His work in the course is both novel and applicable to neuromorphic research, and his engagement with the course material was suitable for a graduate-level course. Accordingly, I support the assignment of a final grade of \textbf{A} for his performance in \textit{ECE 510: Hardware for AI/ML}.

\closing{Sincerely,\\[1ex]
Christof Teuscher}

\end{document}

